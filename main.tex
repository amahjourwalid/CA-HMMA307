\documentclass[unknownkeysallowed]{beamer}
\usepackage[french,english]{babel}
\usepackage{etex} %solve bugs with listings
\usefonttheme[onlymath]{serif} % rounded shapes for maths
\usepackage[utf8]{inputenc}
\usepackage[T1]{fontenc}% accents (for pdf)
\usepackage{lmodern}
\usepackage{xcolor}
\usepackage{verbatim}
\usepackage{dsfont}
\usepackage[thicklines]{cancel}  % to strike out a text / text rayé
\renewcommand{\CancelColor}{\red}
%
\usepackage{graphicx}
\graphicspath{{images/},{prebuiltimages/},{../../sharedimages/}}

\usepackage{amsmath,amssymb,amsfonts,amscd}

\usepackage{appendixnumberbeamer} % to remove counter for appendix.

\usepackage{multicol}
\usepackage{subcaption}
\usepackage{caption}
% \usepackage{algpseudocode}
\hypersetup{colorlinks,linkcolor=,urlcolor=blue}
\usepackage{latexsym}
\usepackage{fancyhdr} % for page layout
\usepackage{stmaryrd} % for llbracket
% \usepackage{transparent}   % for inkscape transparent style (in conflict with tikz)
\usepackage[absolute,overlay]{textpos} % texpos for special fixed positionning of an image
\usepackage{multirow}
\usepackage{blkarray} % block matrix
\usepackage{xparse,soul} %  strikethrough text (barr\'e), \st

\usepackage{mathtools} % to make boxes inside align env.

%%%%%%%%%%%%%%%%%%%%%%%%%%%%%%%%%%%%%%%%%%%%%%%%%%%%%%%%%%%%%%%%%%%%%%%%%%%%%%%
% Movies
\usepackage{animate}
% \usepackage{movie15} % to be prefered?
% \usepackage{media9}


%\usepackage{amsbsy}
%\usepackage{longtable}
%\renewcommand{\labelitemii}{$\star$}
%\renewcommand{\labelitemi}{$\bullet$}


%%%%%%%%%%%%%%%%%%%%%%%%%%%%%%%%%%%%%%%%%%%%%%%%%%%%%%%%%%%%%%%%%%%%%%%%%%%%%%%
%% citation/footenote in beamer
\usepackage[backend=biber,style=verbose,citetracker=true]{biblatex}
\renewcommand{\footnotesize}{\tiny} % size of footnotes
\renewcommand\multicitedelim{\addsemicolon\space} % for footfullcite to have multiple elements separated by commas.

\renewcommand{\footnoterule}{%
\kern -3pt
\hrule width \textwidth height 1.5pt
\kern 3pt
} % control the line before footnotes


%%%%%%%%%%%%%%%%%%%%%%%%%%%%%%%%%%%%%%%%%%%%%%%%%%%%%%%%%%%%%%%%%%%%%%%%%%%%%%%
% footnote setting: (1), (2), etc.
\usepackage{fnpct}
\AdaptNoteOpt\footcite\multfootcite
\renewcommand*{\thefootnote}{(\arabic{footnote})}

% How to solve problem of comas between footfullcite
\AdaptNoteOpt\footfullcite\multfootfullcite


% Configure style for custom doubled line
\newcommand*{\doublerule}{\hrule width \hsize height 1pt \kern 0.5mm \hrule width \hsize height 2pt}

% Configure function to fill line with doubled line
\newcommand\doublerulefill{\leavevmode\leaders\vbox{\hrule width .1pt\kern1pt\hrule}\hfill\kern0pt }


\newcommand{\mytheorem}[2]{
\doublerulefill\ \framebox{\textbf{#1}}\ \doublerulefill
\vspace{0.1cm}

#2

\doublerulefill
}



%%%%%%%%%%%%%%%%%%%%%%%%%%%%%%%%%%%%%%%%%%%%%%%%%%%%%%%%%%%%%%%%%%%%%%%%%%%%%%%
\beamersetuncovermixins{\opaqueness<1>{15}}{\opaqueness<2->{15}}
%%%%%%%%%%%%%%%%%%%%%%%%%%%%%%%%%%%%%%%%%%%%%%%%%%%%%%%%%%%%%%%%%%%%%%%%%%%%%%%


\mode<presentation>
{
\usetheme{default}
\setbeamertemplate{frametitle}
{
  \begin{centering}
    \textbf{\insertframetitle}
    \par
  \end{centering}
}


\setbeamertemplate{navigation symbols}{}
\setbeamertemplate{bibliography item}[triangle]


%%%%%%%%%%%%%%%%%%%%%%%%%%%%%%%%%%%%%%%%%%%%%%%%%%%%%%%%%%%%%%%%%%%%%%%%%%%%%%%
% Colors

%\usecolortheme[named=macouleur]{structure}
\definecolor{macouleur}{cmyk}{0.2,0.45,0.9,0}%{0.05,0.22,0.89,0}
\definecolor{rltred}{rgb}{0.75,0,0}
\definecolor{oneblue}{rgb}{0,0,0.75}
\definecolor{marron}{rgb}{0.64,0.16,0.16}
\definecolor{forestgreen}{rgb}{0.13,0.54,0.13}
\definecolor{verte}{rgb}{0.13,0.74,0.23}
\definecolor{marronchocolat}{cmyk}{0.15,0.32,0.59,0.19}
\definecolor{parme}{cmyk}{0.05,0.22,0.19,0.5}
\definecolor{rltgreen}{rgb}{0.5,0.01,0.9}%violet
\definecolor{oneblue}{rgb}{0,0,0.75}
\definecolor{purple}{rgb}{0.5,0.5,0.98}
\definecolor{purpleone}{rgb}{0.9,0.5,0.5}
\definecolor{dockerblue}{rgb}{0.11,0.56,0.98}
\definecolor{myblue}{rgb}{0,0.2,0.4}%bien
\definecolor{LightBack}{rgb}{0,0.2,0.4}%{0.2, 0.2,0.4}
\definecolor{freeblue}{rgb}{0.25,0.41,0.88}
\usecolortheme{rose}
}

\newcommand{\tbrown}[1]{\textcolor{brown}{#1}}


\usecolortheme[named=marron]{structure}
\setbeamercolor{block title}{use=structure,fg=white,bg=freeblue}
%\setbeamercolor{math text}{fg=LightBack,bg=macouleur}
\setbeamercolor{block body example}{bg=white}
\beamertemplateroundedblocks
\beamertemplateshadowblocks

%%%%%%%%%%%%%%%%%%%%%%%%%%%%%%%%%%%%%%%%%%%%%%%%%%%%%%%%%%%%%%%%%%%%%%%%%%%%%%%
% To add numbering
\setbeamerfont{page number in head/foot}{size=\tiny}
\setbeamertemplate{footline}[frame number]


%\setbeamercolor{structure}{fg=marron,bg=white}


\newcommand{\citer}[2]{\textcolor{brown}{#1}\nocite{#2}}


% new version for algorithm:

% FRENCH VERSION:
% \usepackage[titlenumbered,ruled,noend,french,onelanguage]{algorithm2e}

% ENGLISH VERSION:
\usepackage[titlenumbered,ruled,noend,onelanguage]{algorithm2e}

\newcommand\mycommfont[1]{\footnotesize\ttfamily\textcolor{blue}{#1}}
\SetCommentSty{mycommfont}
\SetEndCharOfAlgoLine{}
\resetcounteronoverlays{algocf} % for overlay aware algorithms.
\renewcommand{\thealgocf}{} % turn of caption numbering
%% old version for algorithm:
%\usepackage[algo2e]{algorithm2e}% http://csweb.ucc.ie/~dongen/LAF/Algorithms.pdf
%
%\usepackage{float}




%%%%%%%%%%%%%%%%%%%%%%%%%%%%%%%%%%%%%%%%%%%%%%%%%%%%%%%%%%%
%%%%% 	        Algorithmes		             %%%%%
%%%%%%%%%%%%%%%%%%%%%%%%%%%%%%%%%%%%%%%%%%%%%%%%%%%%%%%%%%%

\usepackage{fancyvrb}                  % for fancy verbatim
\usepackage{textcomp}
\usepackage[space=true]{accsupp}
% requires the latest version of package accsupp
\newcommand{\copyablespace}{
    \BeginAccSupp{method=hex,unicode,ActualText=00A0}
\ %
    \EndAccSupp{}
}
\usepackage[procnames]{listings}
\lstset{%
language   = Python,%
% basicstyle = \ttfamily\setstretch{1},%
columns    = fullflexible,%
keywordstyle=\color{javared},
firstnumber=100,
frame=shadowbox,
showstringspaces=false,
morekeywords={import,from,class,def,for,while,if,is,in,elif,
else,not,and,or,print,break,continue,return,True,False,None,access,
as,del,except,exec,finally,global,import,lambda,pass,print,raise,try,assert},
keywordstyle={\color{javared}\bfseries},
commentstyle=\color{javagreen}, %vsomb_col white comments
morecomment=[s][\color{javagreen}]{"""}{"""},
upquote=true,
%% style for number
numbers=none,
resetmargins=true,
xleftmargin=10pt,
linewidth= \linewidth,
numberstyle=\tiny,
stepnumber=1,
numbersep=8pt, %
frame=shadowbox,
rulesepcolor=\color{black},
procnamekeys={def,class},
procnamestyle=\color{oneblue}\textbf,
literate={*}{{\char42}}1
         {-}{{\char45}}1
         {\ }{{\copyablespace}}1
}

\lstnewenvironment{Exemplecode}{}{}


\newenvironment{Framecode}[1]
{\begin{frame}[fragile, environment=Framecode]{#1}}
{\end{frame}}


\definecolor{javared}{rgb}{0.6,0,0} % for strings
\definecolor{javagreen}{rgb}{0.25,0.5,0.35} % comments
\definecolor{javapurple}{rgb}{0.5,0,0.35} % keywords
\definecolor{javadocblue}{rgb}{0.25,0.35,0.75} % javadoc
\definecolor{marron}{rgb}{0.64,0.16,0.16}
\definecolor{orange_js}{RGB}{230,159,0}

\newcommand{\mybold}[1]{\textcolor{marron}{\textbf{#1}}}
\newcommand{\cRm}[1]{\textsc{\romannumeral #1}}


\AtBeginDocument{\setlength\abovedisplayskip{0pt}}% remove top align blanks

\input{shortcuts_js}
%\usepackage{./OrganizationFiles/tex/sty/shortcuts_js}
\usepackage{csquotes}

\graphicspath{{./images/}}

\addbibresource{Bibliographie.bib}
\usepackage{enumerate}

\begin{document}


%%%%%%%%%%%%%%%%%%%%%%%%%%%%%%%%%%%%%%%%%%%%%%%%%%%%%%%%%%%%%%%%%%%%%%%%%%%%%%%
%%%%%%%%%%%%%%%%%%%%%%             Headers               %%%%%%%%%%%%%%%%%%%%%%
%%%%%%%%%%%%%%%%%%%%%%%%%%%%%%%%%%%%%%%%%%%%%%%%%%%%%%%%%%%%%%%%%%%%%%%%%%%%%%%



%%%%%%%%%%%%%%%%%%%%%%%%%%%%%%%%%%%%%%%%%%%%%%%%%%%%%%%%%%%%%%%%%%%%%%%%%%%%%%%
\begin{frame}[noframenumbering]
\thispagestyle{empty}
\bigskip
\bigskip
\begin{center}{
\LARGE\color{marron}
\textbf{HMMA 307 : Advanced Linear Modeling}
\textbf{ }\\
\vspace{0.5cm}
}

\color{marron}
\textbf{Chapter 5 : Random Effects Models}
\end{center}

\vspace{0.5cm}

\begin{center}
\textbf{ISKOUNEN SELENA \ AMAHJOUR WALID \ RUDOLF RÖMISCH } \\
\vspace{0.1cm}
\url{https://github.com/selenaiskounen/CA-HMMA307}\\
\vspace{0.5cm}
Université de Montpellier \\
\end{center}

\centering
\includegraphics[width=0.13\textwidth]{Logo.pdf}
\end{frame}

\begin{frame}{Motivation}
We consider cases where we get random samples from large populations. 
Since the goal is to make statements about properties of whole populations and not about observed individuals, it is rather natural to assume random samples. 
Now we elaborate an example with machines. We assume that we assess the quality of produced samples from some machines.

\end{frame}
\begin{frame}{Model}
Our model:

$$
Y_{ij} = \mu + \alpha_i + \epsilon_{ij}
$$
Here we have the following variables and assumptions:
\begin{itemize}
    \item $Y_{ij}$ as the quality of the j-th sample on the i-th machine 
    \item $\mu$ as the global mean
    \item $\alpha_i$ i.i.d.  
    $\sim \mathcal{N}(0,\sigma^2_{\alpha})$ as the effect of the i-th machine
    \item $\epsilon_{ij}$ i.i.d. $\sim \mathcal{N}(0,\sigma^2)$ as the error term for the j-th sample and the i-th machine
\end{itemize}
\end{frame}

\begin{frame}{Specialty}
    The model has strong similarities with the fixed models. But the $\alpha_i's$ are random variables and not fixed unkown parameters.
    With that change the properties of the model will be strongly influenced.
    \begin{itemize}
        \item We get the new parameter $\sigma^2_{\alpha}$
        \item Our model is also called variance components models since we have now two different variances $\sigma^2_{\alpha}$ and $\sigma^2$
    \end{itemize}
\end{frame}

\begin{frame}{Properties}
    We consider some properties for our model:
    \begin{itemize}
        \item The expected value of $Y_{ij}$ is $E[Y_{ij}]=\mu$, since $E[Y_{ij}] = E[\mu + \alpha_i + \epsilon_{ij}] = E[\mu] + E[\alpha_i] + E[\epsilon_{ij}] = \mu$
        \item The variance is $Var(Y_{ij}) = \sigma^2_{\alpha} +\sigma^2$ since
        
        $Var(Y_{ij}) = E[Y_{ij}^2] - E[Y_{ij}]^2
        = E[\mu^2] + E[\mu \alpha_i] + E[\mu \epsilon_{ij}] + E[\alpha_i \mu] + E[\alpha_i \alpha_i] + E[\alpha_i \epsilon_{ij}] + E[\epsilon_{ij} \mu] + E[\epsilon_{ij}\alpha_i] + E[\epsilon_{ij}\epsilon_{ij}] - \mu^2
        = \mu^2 + \mu \underbrace{E[\alpha_i]}_\text{= 0} + \mu \underbrace{E[\epsilon_{ij}]}_\text{= 0} + \underbrace{E[\alpha_i]}_\text{=0}\mu + E[\alpha_i^2] + \underbrace{E[\alpha_i]}_\text{= 0}\underbrace{E[\epsilon_{ij}]}_\text{= 0} + \underbrace{E[\epsilon_{ij}]}_\text{=0}\mu + \underbrace{E[\epsilon_{ij}]}_\text{=0} \underbrace{E[\alpha_i]}_\text{=0} + E[\epsilon_{ij}^2] - \mu^2 
        = E[\alpha_i^2] + E[\epsilon_{ij}^2]
        = \sigma^2_{\alpha} + \sigma^2$
    \end{itemize}
\end{frame}





\end{document}